\section{Exercise 6.3 \emph{``I feel that I know the change that is needed.''} -- Mahatma Gandhi}
\textbf{Problem:} We are given $2m$ numbers satisfying $L_i\leq{}0\leq{}U_i$, $i=1,2,...,m$. Let $\beta$ be an optimal basis for all of the $m$ problems
\begin{equation}
\label{eq: primali}
  \begin{array}{lrcll}
    \min
    & \multicolumn{4}{l}{c'x} \\
    \text{s.t.}
    & Ax=b+\Delta_ie^i;\\
    & x\geq0.
  \end{array}
\end{equation}
for all $\Delta_i$ satisfying $L_i\leq{}\Delta_i\leq{}U_i$. Let’s be clear on what this means: For each $i$ individually, the basis $\beta$ is optimal when the $i$th right-hand side component is changed from $b_i$ to $b_i+\Delta_i$, as long as $\Delta_i$ is in the interval $[L_i,U_i]$. 

The point of this problem is to be able to say something about \textit{simultaneously} changing all of the $b_i$. Prove that we can simultaneously change $b_i$ to
\[
\tilde{b}_i:=b_i+\lambda_i \left\lbrace \begin{array}{c} L_i \\ U_i \end{array}\right\rbrace
\]
where $\lambda_i\geq0$, when $\sum^{m}_{i=1} \lambda_i \leq1$. [Note that in the formula above, for each $i$ we can $i=1$ pick either $L_i$ (a decrease) or $U_i$ (an increase)].

\textbf{Proof:}

Let $D_i=\left\lbrace \begin{array}{c} L_i \\ U_i \end{array}\right\rbrace$, so $\max_{k:h^i_k>0}{-\frac{\overline{b}_k}{h^i_k}}\leq{}D_i\leq{}\min_{k:h^i_k<0}{-\frac{\overline{b}_k}{h^i_k}}$, where  $\overline{b}=A^{-1}_{\beta}b\geq0$, $h^i=A^{-1}_{\beta}e^i$, and $k=1...m$.

Thus, what we need to prove is that the basis $\beta$ is optimal when we simultaneously change $b_i$ to $\tilde{b}_i:=b_i+\lambda_iD_i$, where $\sum^{m}_{i=1}\lambda_i=1$. In the other word, we need to prove that the basis $\beta$ is optimal for (\ref{eq: primalsigmai}), where $\sum^{m}_{i=1}\lambda_i=1$.
\begin{equation}
\label{eq: primalsigmai}
  \begin{array}{lrcll}
    \min
    & \multicolumn{4}{l}{c'x} \\
    \text{s.t.}
    & Ax=b+\sum^{m}_{i=1}\lambda_i{}D_i{}e^i;\\
    & x\geq0.
  \end{array}
\end{equation}

Consider a fixed and optimal basis $\beta$ for (\ref{eq: primalb}). For $\beta$ to be optimal for (\ref{eq: primalsigmai}), we need $A^{-1}_{\beta}(b+\sum^{m}_{i=1}\lambda_i{}D_i{}e^i)=\overline{b}+\sum^{m}_{i=1}\lambda_i{}D_i{}h^i\geq0$.
\begin{equation}
\label{eq: primalb}
  \begin{array}{lrcll}
    \min
    & \multicolumn{4}{l}{c'x} \\
    \text{s.t.}
    & Ax=b;\\
    & x\geq0.
  \end{array}
\end{equation}

For $k=1...m$,

$\overline{b}_k+\sum^{m}_{i=1}\lambda_i{}D_i{}h^i_k$

$=\overline{b}_k+\sum^{m}_{i=1,h^i_k>0}\lambda_i{}D_i{}h^i_k+\sum^{m}_{i=1,h^i_k<0}\lambda_i{}D_i{}h^i_k$

$\geq{}\overline{b}_k+\sum^{m}_{i=1,h^i_k>0}\lambda_i{}(\max_{k':h^i_{k'}>0}{-\frac{\overline{b}_{k'}}{h^i_{k'}}})h^i_k+\sum^{m}_{i=1,h^i_k<0}\lambda_i{}(\min_{k':h^i_{k'}<0}{-\frac{\overline{b}_{k'}}{h^i_{k'}}})h^i_k$

$\geq{}\overline{b}_k+\sum^{m}_{i=1,h^i_k>0}\lambda_i{}(-\frac{\overline{b}_k}{h^i_k})h^i_k+\sum^{m}_{i=1,h^i_k<0}\lambda_i{}(-\frac{\overline{b}_k}{h^i_k})h^i_k$

$=\overline{b}_k+\sum^{m}_{i=1,h^i_k>0}\lambda_i{}(-\overline{b}_k)+\sum^{m}_{i=1,h^i_k<0}\lambda_i{}(-\overline{b}_k)$

$=\overline{b}_k+\sum^{m}_{i=1}\lambda_i{}(-\overline{b}_k)$

$=\overline{b}_k+(-\overline{b}_k)\sum^{m}_{i=1}\lambda_i{}$

$\geq\overline{b}_k+(-\overline{b}_k)$

$=0$

Thus, $\overline{b}+\sum^{m}_{i=1}\lambda_i{}D_i{}h^i\geq0$, and $\beta$ is feasible and hence still optimal for (\ref{eq: primalsigmai}).
\begin{flushright} $\blacksquare$ \end{flushright}