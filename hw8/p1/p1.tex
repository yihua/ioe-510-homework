\section{Exercise 8.1 Task scheduling, continued}
\textbf{Problem:} Consider again the ``task scheduling'' Exercise 2.4. Take the dual of the linear-optimization problem that you formulated. Explain how this dual can be interpreted as a kind of network problem. Using {\tt AMPL}, solve the dual of the example that you created for Exercise
2.4 and interpret the solution.

\textbf{Solution:} Let $p_{i,j}$ represent the precedence matrix, i.e.

\begin{eqnarray}
p_{i,j} =
\begin{cases}
1,   & \text{if job $j$ precedes job $i$}; \\
0,   & \text{otherwise}. \\
\end{cases}
\end{eqnarray}

Then the task scheduling problem can be formulated mathematically as follows:

\[
\tag{P'}
\begin{array}{rrcll}
 \min & t_{n+1}  &  &   & \\
      &  p_{i,j}(t_j + d_j)  &   \leq  & t_i, & \text{for~} p_{i,j} =1; \\
      &  t_0 & = & 0~, & \\
      &  t_i & \geq & 0, & \forall i~.
\end{array}
\]

Let $m$ be the nunmber of pair $(i,j)$ such that $p_{i,j} =1$. Transfering (P') into the standard-form problem (P):

\[
\tag{P}
\begin{array}{rrcll}
 \min & t_{n+1}  &  &   & \\
      &  t_i - t_j - s_k   &   =  & d_j, & \text{for~} p_{i,j} =1; \\
      &  t_0 & = & 0~, & \\
      &  t_i & \geq & 0, & \forall i~, \\
	 &  s_k & \geq & 0, & \forall k~,
\end{array}
\]

which is in standard form of the following

\[
\tag{P}
\begin{array}{rrcl}
 \min & c'x  &      &   \\
      &  Ax  &   =  & b~; \\
      &   x  & \geq & \mathbf{0}~,
\end{array}
\]

where

$$

x = \left(
  \begin{array}{cccc}
    0 & \cdots & 0 & 0 \\
  \end{array}
\right)',~
c = 
\left(
  \begin{array}{c}
    -100 \\
    -300 \\
    0 \\
    0 \\
  \end{array}
\right),~
A = 
\left(
  \begin{array}{cccc}
    1 & 1 & -1 & 0 \\
    10 & 20 & 0 & 1 \\
  \end{array}
\right),~
b =
\left(
  \begin{array}{c}
    10 \\
    150 \\
  \end{array}
\right),~
x =
\left(
  \begin{array}{c}
    x_1 \\
    x_2 \\
    t_1 \\
    t_2 \\
  \end{array}
\right).
$$