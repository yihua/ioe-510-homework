\section{Exercise 8.4 Comparing facility-location formulations}
\textbf{Problem:} We have seen two formulations of the forcing constraints for the uncapacitated facility-location problem. We have a choice of the $mn$ constraints: $−y_{i} + x_{ij} \leq 0$, for $i=1, ..., m$ and $j=1,..., n$, or the $m$ constraints: $−ny_{i}+nx_{ij} \leq 0$, for $i=1,...,m$. Which formulation is stronger? That is, compare (both \textit{computationally} and \textit{analytically}) the strength of the two associated continuous relaxations (i.e., when we relax $y_{i}\in{}\lbrace0,1\rbrace$ to $0\leq{}y_{i}\leq1$, for $i=1,...,m$).  In Appendix A.3, there is AMPL code for trying computational experiments.

\textbf{Solution:} Consider the choice of the $mn$ constraints:

$$S_1: -y_i+x_{ij}\leq0, \text{for } i=1,...,m \text{ and } j = 1,...,n,$$

and the $m$ constraints:

$$S_2: -ny_i+\sum_{j=1}^n x_{ij}\leq0, \text{for } i=1,...,m.$$

Using {\tt AMPL}, we first random generate parameters for the uncapacitated facility-location problem and using this set of parameters to compare the two set of constraints with integer constraint and continuous relaxations. We omit the parameters here because of limited space. 

For integer programming, $S_1$ and $S_2$ give the same optimal solution with total cost of 456 ($y = (1,0,1,0,1,0,1,0,1,0)'$). $x$ are also exactly the same (we don't show $x$ here as its dimension is large). When considering the continuous relaxation, $S_1$ still gives the same optimal solution, while $S_2$ gives a different optimal solution with the total cost of 301.76 ($y = (0.2,0.12,0.16,0.12,0.16,0.04,0.08,0.08,0.04,0)'$), which is lower than the optimal cost for integer programming. From this example, we see that the formulation of $S_1$ is stronger.

We next prove it analytically. $\forall (x,y) \in S_1 $,

$$ -y_i+x_{ij}\leq0, \text{for } i=1,...,m \text{ and } j = 1,...,n,$$

then,

$$\sum_{j=1}^n (-y_i+x_{ij})\leq0, \text{for } i=1,...,m, $$

which is equivalent to

$$-ny_i+\sum_{j=1}^n x_{ij}\leq0, \text{for } i=1,...,m.$$

Thus $ (x,y) \in S_2 $, so $S_1 \subset S_2$. However, $S_2$ is not contained in $S_1$. It's possible to construct an feasible solution $(x,y)$ for $S_2$, such that,

$$y_1 = \frac{1}{n}, x_{1j}=0, \text{for } j = 1,...,n-1, x_{1n} = \frac{2}{n},$$
$$y_2 = 1, x_{2j}=1, \text{for } j = 1,...,n-1, x_{2n} = 1-\frac{2}{n}, $$

and other variables is equal to zero. It is obvious that this $(x,y) \notin S_1$ ($-y_1+ x_{1n} =1/n > 0$). So the formulation of $S_1$ is stronger.