\section{Exercise 8.5 Comparing piecewise-linear formulations}
\textbf{Problem:} We have seen that the adjacency condition for piecewise-linear univariate functions can be modeled by
\begin{equation}
\label{eq: orisystem}
  \begin{array}{lrcll}
    \lambda_1\leq{}y_1; &\\
    \lambda_j\leq{}y_{j-1}+y_j, & \text{for } j=2,...,n-1;\\
    \lambda_n\leq{}y_{n-1}. &
  \end{array}
\end{equation}
An alternative formulation is
\begin{equation}
\label{eq: altsystem}
  \begin{array}{lrcll}
   \sum_{i=1}^{j} y_i\leq{}\sum_{i=1}^{j+1} \lambda_i, & \text{for} j=1,...,n-2;\\
   \sum_{i=j}^{n-1} y_i\leq{}\sum_{i=j}^{n} \lambda_i, & \text{for} j=2,...,n-1.\\
  \end{array}
\end{equation}
Explain why this alternative formulation is valid, and compare its strength to the original formulation, when we relax $y_i\in\lbrace0,1\rbrace$ to $0\leq{}y_i\leq1$, for $i=1,...,n−1$. (Note that for both formulations, we require $\lambda_i\geq0$, for $i=1,...,n$, $\sum_{i=1}^{n} \lambda_i=1$, and
$\sum_{i=1}^{n-1} y_i=1$).

\textbf{Solution:} 

When we have $y_i\in\lbrace0,1\rbrace$, the alternative formulation (\ref{eq: altsystem}) is equivalent to formulation (\ref{eq: orisystem}). In (\ref{eq: orisystem}), the formulation means that if $y_k$ is 1, for some $k$ ($1\leq{}k\leq{}n$), and necessarily all of the other $y_j$ are 0, then only $\lambda_k$ and $\lambda_{k+1}$ can be positive. It is easy to see that in (\ref{eq: altsystem}), the meaning is the same: if $y_k$ is 1, $ \sum_{i=1}^{k} y_i=1\leq{}\sum_{i=1}^{k+1} \lambda_i$, and $\sum_{i=k}^{n-1} y_i=1\leq{}\sum_{i=k}^{n} \lambda_i$, so considering that $\sum_{i=1}^{n} \lambda_i=1$, thus $\sum_{i=1}^{k+1} \lambda_i=\sum_{i=k}^{n} \lambda_i=1$, thus $\lambda_k + \lambda_{k+1}=\sum_{i=1}^{k+1} \lambda_i - (\sum_{i=1}^{n} \lambda_i-\sum_{i=k}^{n} \lambda_i)=1-(1-1)=1$