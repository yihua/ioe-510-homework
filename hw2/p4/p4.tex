\section{Exercise 2.4 Task scheduling}

The objective of the problem is to minimize the start time of task $n+1$, $t_{n+1}$. For the constraints on start time of each task, $$t_1 = 0$$ and $$t_i \geq 0, ~i \in \mathbb{N}, ~2 \leq i \leq n+1.$$ For the precedences between tasks, if task $j$ must be completed before task $i$ can be started, namely $j \in \Psi_i$ ($\Psi_i$ is the set of tasks that must be completed before task $i$ can be started), the start time of task $i$ must be larger than or equal to the start time of task $j$ plus the duration of task $j$, $d_j$, namely, $$t_j + d_j \leq t_i,~i \in \mathbb{N}, ~2 \leq i \leq n+1,~\forall j \in \Psi_i.$$ Thus the task scheduling problem can be formulated mathematically as follows:

\[
\begin{array}{rrcll}
 \min & t_{n+1}  &  &   & \\
      &  t_j + d_j  &   \leq  & t_i, &i \in \mathbb{N}, ~2 \leq i \leq n+1,~\forall j \in \Psi_i; \\
      &  t_1 & = & 0, & \\
      &  t_i & \geq & 0, & i \in \mathbb{N}, ~2 \leq i \leq n+1.
\end{array}
\]

An example of a task-scheduling problem ($n=5$) is given in Table \ref{tab:duration}.

\begin{table*}[!h]
\centering
\small
\begin{tabular}{|c|c|c|c|c|c|c|c|}\hline

$i$ & 0 & 1 & 2 & 3 & 4 & 5 & 6  \\\hline
$d_i$ & 0 & 2 & 1 & 6 & 4 & 2 & 0 \\\hline
$\Psi_i$ & $\varnothing$ & \{0\} & \{1\} & \{2\} & \{1\} & \{3,4\} & \{4,5\} \\\hline
\end{tabular}
\caption{Duration of each task, and precedences between tasks}
\label{tab:duration}
\end{table*}

The model file {\tt ex2.4.mod} in {\tt AMPL} for the task-scheduling problem is as follows:

\bigskip

\hrule

\small
\verbatiminput{ex2.4.mod}
\normalsize

\hrule

\bigskip
